\documentclass[a4paper]{article}
\usepackage[landscape]{geometry}
\usepackage[dvipdfm]{color}

\usepackage{quickrefcard}  

\usepackage{multicol}
\usepackage{amsmath}
\usepackage{amsfonts}

\begin{document}
\begin{multicols*}{3}
\begin{center}
\textbf{Sage Quick Reference (Basic Math)}\\
Peter Jipsen, version 1.1
(w/modification by nu)
\\
latest version at \url{wiki.sagemath.org/quickref}\\
GNU Free Document License, extend for your own use\\
Aim: map standard math notation to Sage commands
\end{center}
\vspace{-2ex}
\sect{Notebook (and commandline)}

Evaluate cell: \shiftEnterkey{}

\EXVAR{com}\tabkey{} tries to complete \sagevar{command}

\EXVAR{command}\EX{?}\tabkey{} shows documentation

\EXVAR{command}\EX{??}\tabkey{} shows source

\EX{a.}\tabkey{} shows all methods for object \sagecommand{a} \quad
(more: \EX{dir(a)})

\EX{search_doc('}\EXVAR{string or regexp}\EX{')} \quad shows links to docs

\EX{search_src('}\EXVAR{string or regexp}\EX{')} \quad shows links to source

\EX{lprint()} \quad toggle \LaTeX{} output mode

\EX{version()} \quad print version of Sage

Insert cell: click on blue line between cells

Delete cell: delete content then backspace

%*********************************************
\sect{Numerical types}

Integers: $\mathbb Z=$ \EX{ZZ} e.g. \EX{-2  -1  0  1  10^100}

Rationals: $\mathbb Q=$ \EX{QQ} e.g. \EX{1/2  1/1000  314/100  -42}

Decimals: $\mathbb R\approx$ \EX{RR} e.g. \EX{.5  0.001  3.14  -42}

Complex: $\mathbb C\approx$ \EX{CC} e.g. \EX{1+i  2.5-3*i}

%*********************************************
\sect{Basic constants and functions}

Constants: $\pi=$ \EX{pi} \quad $e=$ \EX{e} \quad $i=$ \EX{i}
\quad $\infty=$ \EX{oo}

Approximate: \EX{pi.n(digits=18)} $=3.14159265358979324$

%Binary operations: \EX{+  -  *  /  ^|

Functions: 
\EX{sin} 
\EX{cos} 
\EX{tan} 
\EX{sec} 
\EX{csc} 
\EX{cot} 
\EX{sinh} 
\EX{cosh} 
\EX{tanh} 
\EX{sech} 
\EX{csch} 
\EX{coth} 
\EX{log} 
\EX{ln} 
\EX{exp}

$ab=$ \EX{a*b} \quad $\frac a b=$ \EX{a/b} 
\quad 
$a^b=$ \EX{a^b} \quad $\sqrt{x}=$ \EX{sqrt(x)}

$\sqrt[n]{x}=$ \EX{x^(1/n)}
\quad 
$|x|=$ \EX{abs(x)}
\quad 
$\log_b(x)=$ \EX{log(x,b)}

Symbolic variables: e.g.\ 
%\EX{t,u,v,y,z = var('t u v y z')}
\EX{t,u,v,y = var('t u v y')}

Define function: e.g. $f(x)=x^2$ \qquad 
\BreakLineAndIndent
As symbolic function (can integrate, etc): \EX{f(x)=x^2} or 
\BreakLineAndIndent
As Python function: \EX{f=lambda x: x^2} or 
\BreakLineAndIndent[\phantom{\skipin As Python function:}]
\EX{def f(x): return x^2}

%*********************************************
\sect{Operations on expressions}

\EX{factor(...)}\qquad \EX{expand(...)}\qquad \EX{(...).simplify_...}

Symbolic equations: \EX{f(x)==g(x)}

\EX{_} is previous output

\EX{_+a} \quad \EX{_-a} \quad \EX{_*a} \quad \EX{_/a} manipulates equation

Solve $f(x)=g(x)$: \EX{ solve(f(x)==g(x),x)}

\EX{solve([f(x,y)==0, g(x,y)==0], x,y)}

\EX{find_root(f(x), a, b)}\quad find $x\in [a,b]$ s.t. $f(x)\approx 0$

$\displaystyle\sum_{i=k}^n f(i)=$ \EX{sum([f(i) for i in [k..n]])}
%$\sum_{i=k}^n f(i)=$ \EX{sum([f(i) for i in [k..n]])}

$\displaystyle\prod_{i=k}^n f(i)=$ \EX{prod([f(i) for i in [k..n]])}
%$\prod_{i=k}^n f(i)=$ \EX{prod([f(i) for i in [k..n]])}

%*********************************************
\sect{Calculus}

$\displaystyle\lim_{x\to a} f(x)=$ \EX{limit(f(x), x=a)}

$\displaystyle\lim_{x\to a^-} f(x)=$ \EX{limit(f(x), x=a, dir='minus')}

$\displaystyle\lim_{x\to a^+} f(x)=$ \EX{limit(f(x), x=a, dir='plus')}

$\frac{d}{dx}(f(x))=$ \EX{diff(f(x),x)}

$\frac{\partial}{\partial x}(f(x,y))=$ \EX{diff(f(x,y),x)}

\EX{diff} $=$ \EX{differentiate} $=$ \EX{derivative}

$\int f(x)dx=$ \EX{integral(f(x),x)}

\EX{integral} = \EX{integrate}

$\int_a^b f(x)dx=$ \EX{integral(f(x),x,a,b)}

Taylor polynomial, deg $n$ about $a$: \EX{taylor(f(x),x,a,n)} 

%*********************************************
\sect{2d graphics}

\EXtt{line([(\EXvar{x_1},\EXvar{y_1}),\EXvar{\ldots},(\EXvar{x_n},\EXvar{y_n})],\EXVAR{options})}

\EXtt{polygon([(\EXvar{x_1},\EXvar{y_1}),\EXvar{\ldots},(\EXvar{x_n},\EXvar{y_n})],\EXVAR{options})}

\EXtt{circle((\EXvar{x},\EXvar{y}),\EXvar{r},\EXVAR{options})}

\EXtt{text("txt",(\EXvar{x},\EXvar{y}),\EXVAR{options})}

\sagevar{options} as in \sagecommand{plot.options}, 
\BreakLineAndFulshLeft
e.g. 
\sagecommand{thickness=}\sagevar{pixel},
\sagecommandtt{rgbcolor=($r$,$g$,$b$)},
\sagecommandtt{hue=$h$}, 
\BreakLineAndFulshLeft
where $0\le r,b,g,h\le 1$

use option \sagecommand{figsize=[w,h]} to adjust aspect ratio

\EXtt{plot(f(\EXvar{x}),\EXvar{x_{\rm min}},\EXvar{x_{\rm max}},\EXVAR{options})}

\EX{parametric_plot}\EXtt{((f(\EXvar{t}),g(\EXvar{t})),\EXvar{t_{\rm min}},\EXvar{t_{\rm max}},\EXVAR{options})}

\EX{polar_plot}\EXtt{(f(\EXvar{t}),\EXvar{t_{\rm min}},\EXvar{t_{\rm max}},\EXVAR{options})}

combine graphs: 
\EX{circle((1,1),1)+line([(0,0),(2,2)])}

\EX{animate(}\EXVAR{list of graphics objects
}\EX{,}\EXVAR{options}\EX{).show(delay=20)}

%*********************************************
\sect{3d graphics}

\EXtt{line3d([(\EXvar{x_1},\EXvar{y_1},\EXvar{z_1}),\EXvar{\ldots},(\EXvar{x_n},\EXvar{y_n},\EXvar{z_n})],\EXVAR{options})}

\EXtt{sphere((\EXvar{x},\EXvar{y},\EXvar{z}),\EXvar{r},\EXVAR{options})}

\EXtt{tetrahedron((\EXvar{x},\EXvar{y},\EXvar{z}),\EXVAR{size},\EXVAR{options})}

\EXtt{cube((\EXvar{x},\EXvar{y},\EXvar{z}),\EXVAR{size},\EXVAR{options})}

\EXtt{octahedron((\EXvar{x},\EXvar{y},\EXvar{z}),\EXVAR{size},\EXVAR{options})}

\EXtt{dodecahedron((\EXvar{x},\EXvar{y},\EXvar{z}),\EXVAR{size},\EXVAR{options})}

\EXtt{icosahedron((\EXvar{x},\EXvar{y},\EXvar{z}),\EXVAR{size},\EXVAR{options})}


\sagevar{options} e.g. \sagecommand{aspect_ratio=[1,1,1] color='red' opacity}


\EXtt{plot3d(f(\EXvar{x},\EXvar{y}),[\EXvar{x_{\rm b}},\EXvar{x_{\rm e}}],[\EXvar{y_{\rm b}},\EXvar{y_{\rm e}}],\EXVAR{options})}

add option \sagecommand{plot_points}\sagecommandtt{=[$m,n$]} or use \sagecommand{plot3d_adaptive}

\EX{parametric_plot3d}\EXtt{((f(\EXvar{t}),g(\EXvar{t}),h(\EXvar{t})),[\EXvar{t_{\rm b}},\EXvar{t_{\rm e}}],\EXVAR{options})}

\EX{parametric_plot3d}\EXtt{((f(\EXvar{u},\EXvar{v}),g(\EXvar{u},\EXvar{v}),h(\EXvar{u},\EXvar{v})),}
\BreakLineAndFulshLeft
\EXtt{[\EXvar{u_{\rm b}},\EXvar{u_{\rm e}}],[\EXvar{v_{\rm b}},\EXvar{v_{\rm e}}],\EXVAR{options})}

use \EX{+} to combine graphics objects

%*********************************************
\sect{Discrete math}

$\lfloor x\rfloor=$ \EX{floor(x)} 
\quad 
$\lceil x\rceil=$ \EX{ceil(x)}

Remainder of $n$ divided by $k=$ {\ex\verb|n%k|} \quad\, $k|n$ iff {\ex\verb|n%k==0|}

$n!=$ \EX{factorial(n)} \qquad
${x\choose m}=$ \EX{binomial(x,m)}

$\phi=$ \EX{golden_ratio} \qquad $\phi(n)=$ \EX{euler_phi(n)}

Strings: e.g. \ \EX{s = 'Hello'} = \EX{"Hello"} = \EX{""+"He"+'llo'}

\texttt{s[0]='H' \quad s[-1]='o' \quad s[1:3]='el' \quad s[3:]='lo'}

Lists: e.g. \ \EX{[1,'Hello',x]} = \EX{[]+[1,'Hello']+[x]}

Tuples: e.g. \ \EX{(1,'Hello',x)} \quad (immutable)

Sets: e.g. \ $\{1,2,1,a\}=$ \EX{Set([1,2,1,'a'])} \ ($=\{1,2,a\}$)

List comprehension $\approx$ set builder notation, e.g.
\BreakLineAndFulshLeft
$\{f(x):x\in X, x>0\}=$ 
\EX{Set([f(x) for x in X if x>0])}

%*********************************************
\sect{Linear algebra}

$\begin{pmatrix}1\\2\end{pmatrix}=$ \EX{vector([1,2])}

$\begin{pmatrix}1&2\\3&4\end{pmatrix}=$ \EX{matrix([[1,2],[3,4]])}

$\left|\begin{matrix}1&2\\3&4\end{matrix}\right|=$
\EX{det(matrix([[1,2],[3,4]]))}

$Av=$ \EX{A*v} \quad $A^{-1}=$ \EX{A^-1} \quad $A^t=$ \EX{A.transpose()}

methods: \EX{nrows() ncols() nullity() rank() trace()}...

%*********************************************
\sect{Sage modules and packages}

\EX{from} \EXVAR{module_name} \EX{import *} \qquad (many preloaded)

e.g. 
\sagecommand{calculus} 
\sagecommand{coding}
\sagecommand{combinat} 
\sagecommand{crypto} 
\sagecommand{functions} 
\sagecommand{games}
\sagecommand{geometry} 
\sagecommand{graphs} 
\sagecommand{groups} 
\sagecommand{logic} 
\sagecommand{matrix} 
\sagecommand{numerical} 
\sagecommand{plot} 
\sagecommand{probability} 
\sagecommand{rings} 
\sagecommand{sets} 
\sagecommand{stats}

\EX{sage.}\EXVAR{module_name}\EX{.all.}\tabkey{} shows exported commands

Std packages: Maxima GP/PARI GAP Singular R Shell...

Opt packages: Biopython Fricas(Axiom) Gnuplot Kash...

{\ex\verb!%!}\EXVAR{package_name} then use package command syntax

\EX{time} \EXVAR{command} \quad to show timing information

\end{multicols*}

\end{document}
